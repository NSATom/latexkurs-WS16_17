%%%names
% XeTeX engine
% use KOMA class for cool presets like A4
\documentclass{scrartcl}

% load default packages for german and unicode support
\usepackage{polyglossia}
\setmainlanguage{german}
\usepackage{xltxtra}
\usepackage{amsmath}
\usepackage{txfonts} % for integrals

% start with the actual content
\begin{document}

Im folgenden die Maxwell-Gleichungen:
\begin{align}
&\nabla \times \vec{E} = - \frac{\partial \vec{B}}{\partial t}
\\
&\nabla \times \vec{B} = \vec{j} + \frac{\partial \vec{E}}{\partial t}
\\
&\nabla \cdot \vec{E} = \rho \vphantom{\frac{\partial \vec{E}}{\partial t}} % \vphantom-command for same vertical distance between lines
\\
&\nabla \cdot \vec{B} = 0 \vphantom{\frac{\partial \vec{E}}{\partial t}}
\end{align}


Das folgende Integral kann jetzt dargestellt werden:
\begin{equation}
\oint \kern-0.45em \oiintctrclockwise \kern-1.3em \oiintclockwise  \kern-1.1em \vec{\vphantom{\oint}}
% \kern verschiebt die Integrale ineinander/ den Vektorpfeil über die Integrale
% \vec{\vphantom{\oint}} erzeugt einen Vektorpfeil auf Höhe über dem Integral
\end{equation}  


Zwei Formeln aus dem tractatus philosophicus:
\begin{equation}
\text{tractatus philosophicus, Satz 6. }
  \left\{
    \begin{matrix}
       
       & 03:  & [ o, \xi, \xi + 1] \\
       & 231: & 1 + 1 + 1 + 1 = (1+1) + (1+1) 
    \end{matrix}
  \right.
\end{equation}
	
	% end of written content
\end{document}