% !TEX program = xelatex
% !TEX encoding = UTF-8 Unicode
% !TEX spellcheck = de-DE

% use KOMA class for cool presets like A4
\documentclass{scrartcl}

% load default packages for german and unicode support
\usepackage{polyglossia}
\setmainlanguage{german}
\usepackage{xltxtra}
\usepackage[backend=biber, citestyle=authoryear, bibstyle=authoryear]{biblatex}
\addbibresource{bibfile.bib}
% set main font to our beloved libertine fonts
\setmainfont{Linux Libertine O}

% start with the actual content
\begin{document}
	
Physik ist nicht lustig, das sollte jedem klar sein, der es studiert. Aber sie ist interessant.\footnote{\cite[S. 2]{Demtr}} Vor allem theoretische Physik.\footnote{\cite[S. 5]{Nolt}}
Aber man meistert das Studium nicht ohne ein wenig Anstrengung.\footnote{\cite[S.3]{Bart}}
\printbibliography
% end of written content
\end{document}

% vim: ft=tex et sw=2 tw=80:
