\section{Matrizen}
\subsection{Der ultimative Matrix-Trick}
In der linearen Algebra wird gerne die Aufgabe gestellt, eine Projektions-, Dreh- oder Spiegelmatrix $A$ mit vorgegebenen Eigenschaften aufzustellen. Dazu stellt man dann meist ein mehr oder weniger kompliziertes Gleichungssystem auf, das es dann zu lösen gilt. Es gibt aber einen viel einfacheren Weg.\\
Für den Fall, dass man \textit{genau} weiß, wohin die Einheitsvektoren abgebildet werden, ist die Lösung sehr einfach. Schauen wir erstmal, was passiert, wenn wir eine beliebige Matrix $A$ auf die Einheitsvektoren anwenden:

\begin{align}
A \,\vec{e_{1}} = 
\begin{pmatrix}
a_{11} & a_{12} & a_{13}\\
a_{21} & a_{22} & a_{23}\\
a_{31} & a_{32} & a_{33}
\end{pmatrix}
\begin{pmatrix}
1\\
0\\
0
\end{pmatrix}
=
\begin{pmatrix}
a_{11}\\
a_{21}\\
a_{31}
\end{pmatrix}\\
A \,\vec{e_{2}} = 
\begin{pmatrix}
a_{11} & a_{12} & a_{13}\\
a_{21} & a_{22} & a_{23}\\
a_{31} & a_{32} & a_{33}
\end{pmatrix}
\begin{pmatrix}
0\\
1\\
0
\end{pmatrix}
=
\begin{pmatrix}
a_{12}\\
a_{22}\\
a_{32}
\end{pmatrix}\\
A \,\vec{e_{3}} = 
\begin{pmatrix}
a_{11} & a_{12} & a_{13}\\
a_{21} & a_{22} & a_{23}\\
a_{31} & a_{32} & a_{33}
\end{pmatrix}
\begin{pmatrix}
0\\
0\\
1
\end{pmatrix}
=
\begin{pmatrix}
a_{13}\\
a_{23}\\
a_{33}
\end{pmatrix}
\end{align}\\
Die drei Ergebnisse sind genau die Einträge der Matrix, wobei die erste Spalte das Abbild der x-Achse, die zweite Spalte das Abbild der y-Achse und die dritte Spalte das Abbild der z-Achse darstellt.\\
Als Beispiel suchen wir die Matrix $A$, die alle Punkte um 90° gegen den Uhrzeigersinn (im mathematisch positiven Sinn) um die z-Achse dreht. Nun überlegen wir uns, was mit den Einheitsvektoren passiert:
\begin{align}
A\,
\begin{pmatrix}
1\\
0\\
0
\end{pmatrix}
&=
\begin{pmatrix}
0\\
1\\
0
\end{pmatrix}\\
A\,
\begin{pmatrix}
0\\
1\\
0
\end{pmatrix}
&=
\begin{pmatrix}
-1\\
0\\
0
\end{pmatrix}\\
A\,
\begin{pmatrix}
0\\
0\\
1
\end{pmatrix}
&=
\begin{pmatrix}
0\\
0\\
1
\end{pmatrix}
\end{align}\\
Die gesuchte Matrix $A$ ist also:

\begin{align}
A = 
\begin{pmatrix}
0&-1&0\\
1&0&0\\
0&0&1
\end{pmatrix}
\end{align}