% !TEX program = xelatex
% !TEX encoding = UTF-8 Unicode
% !TEX spellcheck = de_DE

\documentclass{scrreprt}

\usepackage{blindtext,booktabs,color,nicefrac,polyglossia,setspace,xltxtra,yfonts,hyperref,}

\setmainlanguage{german}

\setromanfont[Mapping=tex-text]{Linux Libertine O}
\setsansfont[Mapping=tex-text]{Linux Biolinum O}
\usepackage{amsmath}
\usepackage{tikz}
\usepackage{pgfplots}
\onehalfspacing
\usepackage{subcaption} %für Abb. 4
\usepackage{imakeidx}
\makeindex

\begin{document}
\author{Rüd, Schnell, Breidenbach}
\title{Uebung 7}
\date{\today}
\maketitle
\tableofcontents
\newpage
\subsection*{Abstract}
Dieses Dokument dient der Übung des Satzes von umfangreichen Projekten in \LaTeX\index{LaTeX@\LaTeX}. Es gehört zum siebten Übungszettel des \LaTeX-Kurses im Wintersemester 2016\,/\,17. Inhaltlich hat es so ziemlich nichts zu bieten, es könnte aber interessant sein, sich den zugehörigen Sourcecode mal anzusehen, da er eine Menge interessanter \LaTeX-Kommandos enthält.



\setchapterpreamble[o]{\dictum[W. Busch]{Stets findet Überraschung statt. Da, wo man's nicht erwartet hat.}}

\chapter{Einleitung}

\blindtext$\sin(x)\cdot\cos(x) = -\nicefrac{1}{2} \cos(2x)$\footnote{Man baechte auch, dass $\sin(x\pm y) = \sin(x)\cos(y) \pm \cos(x)\sin(y)$}

\begin{table}[h]
  \centering
  \begin{tabular}{ccc}
    \toprule
    a & b & c\\
    d & e & f\\
    g & h & i\\
    \bottomrule
  \end{tabular}
  \caption[Erste Tabelle]{Die erste Tabelle}
  \label{tabelle1}
\end{table}
\setchapterpreamble[o]{\dictum[Cato]{\textsc{Nullus est liber tam malus, ut non aliqua parte prosit.}}}

\index{Blinddokument|(}
\blinddocument
\index{Blinddokument|)}

\begin{figure}[h]
  \centering
  \fbox{I am a picture!}
  \caption{Ein Bild, das die Aussage des Textes unterstreicht.}
  \label{statement}
\end{figure}

\setchapterpreamble[o]{\dictum[U.\,R. Heber]{Ein schlauer Spruch bereichert den Kapitelanfang.}}

\chapter{Ein weiteres Kapitel}


\begin{figure}[h]
	\centering
	\fbox{I am a picture, introducing this chapter!}
	\caption{Bildunterschrift}
	\label{introduction}
\end{figure}

\index{Blindtext}
\blindtext\\
\blindtext\\
\blindtext\\
\blindtext\footnote{Man beachte auch, dass $\sin(x\pm y) = \sin(x)\cos(y) \pm \cos(x)\sin(y)$}

\begin{table}[h]
  \centering
  \begin{tabular}{ccc}
  \toprule
  eins & zwei & drei\\
  vier & fünf & sechs\\
  sieben & acht & neun\\
  \bottomrule
  \end{tabular}
  \caption{Eine Tabelle mit neun Einträgen}
  \label{TabNeunEinträge} % weil nach dem Einfügen einer weiteren Tabelle die Nummerierung floeten geht
\end{table}

\setchapterpreamble[o]{\dictum[F. Halm]{\hspace*{2em}\textfrak{Ruhe bleibt den Leichen;\\ Der Lebende tauch' frisch ins: Lebens:meer.}}}


\chapter{Und noch ein weiteres Kapitel}
\index{Blindtext}
\Blindtext Der \verb$\Bindtext$-Befehl\index{Blindtext} ist eine nette Sache, wenn man in \LaTeX\index{LaTeX@\LaTeX} sehen will, wie ein Dokument mit viel Inhalt aussieht, ohne, dass man Inhalt hat.\\
\\
\begin{figure}[h] % keine extra Seite notwendig
	\begin{subfigure}{.5\textwidth}
		\centering
		\caption{Beispiel zu diesem Kapitel}
		\fbox{I am a picture!}
		\label{example}
	\end{subfigure}
	\begin{subfigure}{.5\textwidth}
		\centering
		\caption{Veranschaulichung der Aussage}
		\fbox{I am a picture!}
		\label{illustration}
	\end{subfigure}
	
	\begin{subfigure}{.5\textwidth}
		\centering
		\caption{Detailansicht}
		\fbox{I am a picture!}
		\label{detail}
	\end{subfigure}
	\begin{subfigure}{.5\textwidth}
		\centering
		\caption{Visualisierung des Ergebnisses}
		\fbox{I am a picture!}
		\label{visualization}
	\end{subfigure}
\end{figure}


\chapter{Extra Kram}
\section{Symmetrien}
\subsection{Symmetrische und antisymmetrische Funktionen}

Unser erstes Hilfsmittel, um uns das Leben leichter zu machen, sind Symmetrien. Wir betrachten hier nur \textbf{Spiegelsymmetrie zur y-Achse} (symmetrisch) und \textbf{Punktsymmetrie zum Ursprung} (antisymmetrisch). Symmetrische Funktionen erfüllen die Bedingung
\begin{align}
f(x) = f(-x)
\label{sym}
\end{align}

Antisymmetrische Funktionen erfüllen die Bedingung:
\begin{align}
g(x) = -g(-x)
\label{asym}
\end{align}
Für dieses Kapitel wird $f(x)$ immer eine symmetrische und $g(x)$ immer eine antisymmetrische Funktion darstellen. Alle folgenden Überlegungen basieren auf den Gleichungen \ref{sym} und \ref{asym}. \\
\\
\begin{figure}[h]
\begin{tikzpicture}
\begin{axis}[
domain=-2:2,
axis x line=bottom, % no box around the plot, only x and y axis
axis y line=left, % the * suppresses the arrow tips
enlargelimits=upper] % extend the axes a bit to the right and top
\addplot[no markers, blue]{x}
node[pos=0.7,pin=135:{\color{red}$g(x) = x$}] {};
\end{axis}
\end{tikzpicture}
\begin{tikzpicture}
\begin{axis}[
domain=-2:2,
axis x line=bottom, % no box around the plot, only x and y axis
axis y line=left, % the * suppresses the arrow tips
enlargelimits=upper] % extend the axes a bit to the right and top
\addplot[no markers, blue]{x*x}
node[pos=0.3,pin=30:{\color{red}$f(x) = x^{2}$}] {};
\end{axis}
\end{tikzpicture}
\caption{Antisymmetrische und Symmetrische Funktion}
\label{SymAndAnti}
\end{figure}\\
\\
Da Flächen \textbf{unter} der x-Achse ($y=0$) \textbf{negative} Beiträge geben, lassen sich einige Integrale besonders einfach lösen oder zumindest stark vereinfachen. Integriert man eine antisymmetrische Funktion $g(x)$ über \textbf{symmetrische Integrationsgrenzen}, d.h. von $-a$ bis $a$, \,ist das Ergebnis immer \textbf{null}.\\
\begin{align}
\int_{-a}^a g(x) \, dx = G(a) - G(-a) = 0
\end{align}\\
Hier nutzen wir aus, dass $G(a)=G(-a)$ ist, die Stammfunktion $G(x)$ einer antisymmetrischen Funktion $g(x)$ also eine symmetrische Funktion ist.
\footnote{Siehe Beweis \ref{Stammfkt}.}\\
Integriert man eine symmetrische Funktion $f(x)$ über symmetrische Grenzen, sind die Flächeninhalte für positive und negative x-Werte gleich:\\
\begin{align}
\int_{-a}^a f(x) \, dx &= F(a) - F(-a) = F(a) -(-F(a))\\
&= 2\,F(a) = 2 \int_{0}^a f(x) \, dx
\end{align}\\
Hier nutzen wir aus, dass die Stammfunktion $F(x)$ einer symmetrischen Funktion $f(x)$ eine antisymmetrische Funktion ist.\footnote{Siehe Beweis \ref{Stammfkt}.}\\
Beim Integrieren einer symmetrischen Funktion haben wir also noch nicht so viel gewonnen. Müssen wir aber eine antisymmetrische Funktion über symmetrische Grenzen integrieren, wissen wir das Ergebnis, ohne eine Stammfunktion bilden zu müssen.\\
\textbf{Wichtig:} Beide Tricks funktionieren nur, wenn die Integrationsgrenzen, wie schon oft gesagt,  symmetrisch sind.

\subsection{Beispiele für (anti)symmetrische Funktionen}
\fbox{\begin{minipage}{\textwidth}
		\centering
	Alle Polynome mit ausschließlich geraden Exponenten sind symmetrisch.\\
	Alle Polynome mit ausschließlich ungeraden Exponenten sind antisymmetrisch.
\end{minipage}}

 \begin{align*}
 f(x) = \dots + ax^{-4} + bx^{-2} + c + dx^{2} + ex^{4} + fx^{6}\dots\\
 g(x) = \dots + ax^{-3} + bx^{-1} + cx + dx^{3} + ex^{5}+\dots
 \end{align*}\\
 Außerdem:
 \begin{align*}
 f(x) = cos(x)\\
 g(x) = sin(x)
 \end{align*}
 
 Die meisten anderen Funktionen weisen keine Symmetrien auf.
 
 \subsection{Multiplikation (anti)symmetrischer Funktionen}
 
 Was passiert eigentlich, wenn eine (anti)symmetrische Funktion mit einer (anti)symmetrischen Funktion multipliziert wird?\\
 \\
 Seien $f(x)$,$v(x)$ symmetrisch und $g(x)$,$k(x)$ antisymmetrisch:\\
 \begin{align*}
 u(x)=f(x) v(x) = f(-x) v(-x) = u(-x)\\
 \\
 u(x)=f(x) g(x) = f(-x) (-g(-x)) = - f(-x) g(-x) = -u(-x)\\
 \\
 u(x)=g(x) k(x) = (-f(-x)) (-g(-x)) = f(-x) g(-x) = u(-x)
 \end{align*}\\
 Es zeigt sich, dass sich (anti)symmetrische Funktion bei Multiplikation wie Multiplikationen von $1$ und $-1$ verhalten.\\
 \begin{itemize}{}{}
 	 \item Multipliziert man zwei symmetrische Funktionen, bleibt das Produkt symmetrisch.
 	 \item Multipliziert man eine symmetrische und eine antisymmetrische Funktion, ist das Produkt antisymmetrisch.
 	 \item Multipliziert man zwei antisymmetrische Funktionen, ist das Produkt symmetrisch.
 \end{itemize}

\subsection{Addition (anti)symmetrischer Funktionen}
\textit{"Und wenn ich eine Summe aus (anti)symmetrischen Funktionen integrieren muss?"}\\

\begin{align*}
\int_{-a}^{a} f(x) + g(x) dx = \int_{-a}^{a} f(x) dx + \int_{-a}^{a} g(x) dx  
\end{align*}\\
No more words needed\dots\\
\subsection{Rechenbeispiele}
\begin{align*}
\int_{-a}^{a} x^{2} sin(x) dx &= 0 & \int_{-a}^{a} x cos(x) dx &= 0\\
\int_{-a}^{a} \frac{sin(x)}{x^{2}} dx &= 0 & \int_{-a}^{a} \frac{cos(x)}{x} dx &= 0\\
\end{align*}\\
In allen vier Beispielen wurde jeweils eine symmetrische mit einer antisymmetrischen Funktion multipliziert und somit eine antisymmetrische Funktion geschaffen.\\
\\
\textbf{Anmerkung:} Irgendwie hab ich das Gefühl, dass die Lehrer euch nie eine solche Aufgabe geben werden. Da ginge doch der ganze Spaß am Integrieren verloren. Deshalb jetzt die wirklichen Tricks!
\section{Matrizen}
\subsection{Der ultimative Matrix-Trick}
In der linearen Algebra wird gerne die Aufgabe gestellt, eine Projektions-, Dreh- oder Spiegelmatrix $A$ mit vorgegebenen Eigenschaften aufzustellen. Dazu stellt man dann meist ein mehr oder weniger kompliziertes Gleichungssystem auf, das es dann zu lösen gilt. Es gibt aber einen viel einfacheren Weg.\\
Für den Fall, dass man \textit{genau} weiß, wohin die Einheitsvektoren abgebildet werden, ist die Lösung sehr einfach. Schauen wir erstmal, was passiert, wenn wir eine beliebige Matrix $A$ auf die Einheitsvektoren anwenden:

\begin{align}
A \,\vec{e_{1}} = 
\begin{pmatrix}
a_{11} & a_{12} & a_{13}\\
a_{21} & a_{22} & a_{23}\\
a_{31} & a_{32} & a_{33}
\end{pmatrix}
\begin{pmatrix}
1\\
0\\
0
\end{pmatrix}
=
\begin{pmatrix}
a_{11}\\
a_{21}\\
a_{31}
\end{pmatrix}\\
A \,\vec{e_{2}} = 
\begin{pmatrix}
a_{11} & a_{12} & a_{13}\\
a_{21} & a_{22} & a_{23}\\
a_{31} & a_{32} & a_{33}
\end{pmatrix}
\begin{pmatrix}
0\\
1\\
0
\end{pmatrix}
=
\begin{pmatrix}
a_{12}\\
a_{22}\\
a_{32}
\end{pmatrix}\\
A \,\vec{e_{3}} = 
\begin{pmatrix}
a_{11} & a_{12} & a_{13}\\
a_{21} & a_{22} & a_{23}\\
a_{31} & a_{32} & a_{33}
\end{pmatrix}
\begin{pmatrix}
0\\
0\\
1
\end{pmatrix}
=
\begin{pmatrix}
a_{13}\\
a_{23}\\
a_{33}
\end{pmatrix}
\end{align}\\
Die drei Ergebnisse sind genau die Einträge der Matrix, wobei die erste Spalte das Abbild der x-Achse, die zweite Spalte das Abbild der y-Achse und die dritte Spalte das Abbild der z-Achse darstellt.\\
Als Beispiel suchen wir die Matrix $A$, die alle Punkte um 90° gegen den Uhrzeigersinn (im mathematisch positiven Sinn) um die z-Achse dreht. Nun überlegen wir uns, was mit den Einheitsvektoren passiert:
\begin{align}
A\,
\begin{pmatrix}
1\\
0\\
0
\end{pmatrix}
&=
\begin{pmatrix}
0\\
1\\
0
\end{pmatrix}\\
A\,
\begin{pmatrix}
0\\
1\\
0
\end{pmatrix}
&=
\begin{pmatrix}
-1\\
0\\
0
\end{pmatrix}\\
A\,
\begin{pmatrix}
0\\
0\\
1
\end{pmatrix}
&=
\begin{pmatrix}
0\\
0\\
1
\end{pmatrix}
\end{align}\\
Die gesuchte Matrix $A$ ist also:

\begin{align}
A = 
\begin{pmatrix}
0&-1&0\\
1&0&0\\
0&0&1
\end{pmatrix}
\end{align}
\printindex
\renewcommand{\listfigurename}{Liste der Abbildungen}
\renewcommand{\listtablename}{Liste der Tabellen}
\listoffigures
\listoftables
\end{document}
