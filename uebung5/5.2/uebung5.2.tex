% !TEX program = pdflatex
% !TEX encoding = UTF-8 Unicode
% !TEX spellcheck = de-DE

% use KOMA class for cool presets like A4
\documentclass{scrartcl}

% load default packages for german and unicode support
\usepackage[ngerman]{babel}
\usepackage[T1]{fontenc}
\usepackage[utf8]{inputenc}

% load font to use (default linux libertine because it is a neat one
\usepackage{libertine}
\usepackage{libertinust1math}
\usepackage{tikz}

% start with the actual content
\begin{document}
	
\begin{tikzpicture}
[level distance=15mm, 
 level 1/.style={sibling distance=40mm},
 fach/.style={grow=down,xshift=1em,anchor=west,
 	edge from parent path={(\tikzparentnode.south) |- (\tikzchildnode.west)}},
 first/.style={level distance=6ex},
 second/.style={level distance=12ex},
 third/.style={level distance=18ex},
 fourth/.style={level distance=24ex}]

\node {Vorlesungen}
child {node {Informatik}
	child[fach,first] {node {Praktische}}
	child[fach,second] {node {Algorithmen}}
	child[fach,third] {node {Datenstrukturen}}
	child[fach,fourth] {node {Theoretische}}}
child {node {Physik}
	child[fach,first] {node {Experimental}}
	child[fach,second] {node {Theoretische}}}
child {node {Mathematik}
	child[fach,first] {node {Analysis}}
	child[fach,second] {node {Lineare Algebra}}
};
\end{tikzpicture}


% end of written content
\end{document}

% vim: ft=tex et sw=2 tw=80:
