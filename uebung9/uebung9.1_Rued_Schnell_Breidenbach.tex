% !TEX program = xelatex
% !TEX encoding = UTF-8 Unicode
% !TEX spellcheck = de-DE

% use KOMA class for cool presets like A4
\documentclass{beamer}

% load default packages for german and unicode support
\usepackage{polyglossia}
\setmainlanguage{german}
\usepackage{xltxtra}

% set main font to our beloved libertine fonts
\setmainfont{Linux Libertine O}
\usetheme{Rochester} %Theme
\usecolortheme{whale}
\usepackage{showexpl}
\usepackage{showexpl,xcolor}
\definecolor{light-gray}{gray}{0.95}

% start with the actual content
\begin{document}
\title{Verwendungsbeispiele in \LaTeX}
\author{Hein-Erik Schnell, Sören Breidenbach, Thomas Rüd}

\frame{\titlepage}

\begin{frame}{Übersicht}
	\tableofcontents
\end{frame}

\section{Präsentationen}
\subsection{Was ist wichtig?}

\begin{frame}{Was kann \LaTeX?}{Präsentationen}
\begin{itemize}
	\item[]<1-> \LaTeX bietet die Möglichkeit, Präsentationen zu erstellen.
	\item[]<1->
	\item[]<2-> Dabei ist wichtig:
		\begin{itemize}
		\item[]<3->
		\item<3-> Nicht zu viel Text
		\item<4-> übersichtlich gestalten
		\item<5-> Rechtschreibung
	    \end{itemize}	
\end{itemize}	
\end{frame}

\section{Grafiken}
\subsection{Paket graphicx}

\begin{frame}{Was kann \LaTeX?}{Grafiken}
	\begin{itemize}
		\item[]<1-> Mit dem Paket \textcolor{blue}{graphicx} kann man Bilder und Grafiken einfügen
		\item[]<1->
		\item[]<2-> Was kann \textcolor{blue}{graphicx}?
		\begin{itemize}
			\item[]<3->
			\item<3-> Größe der Bilder einstellen
			\item<4-> Bilder rotieren
			\item<5-> Beschriften
		\end{itemize}
	\end{itemize}
	
\end{frame}
\subsection{Source-Code-Beispiel}
\begin{frame}[fragile]{Was kann \LaTeX?}{Code-Beispiel}
	\begin{block}{Code-Beispiel}
		\begin{verbatim}
\usepackage{graphicx}
\begin{document}
\includegraphics[width=2cm]{filename.jpg}
\caption{Bild 1}
\end{document}
\end{verbatim}
	\end{block}
\end{frame}

% end of written content
\end{document}

% vim: ft=tex et sw=2 tw=80:
