% !TEX program = xelatex
% !TEX encoding = UTF-8 Unicode
% !TEX spellcheck = de-DE

% use KOMA class for cool presets like A4
\documentclass{beamer}

% load default packages for german and unicode support
\usepackage{polyglossia}
\setmainlanguage{german}
\usepackage{xltxtra}

% set main font to our beloved libertine fonts
\setmainfont{Linux Libertine O}
\setmonofont[Scale=.95,AutoFakeSlant]{Inconsolata}
\usetheme{Rochester} %Theme
\usecolortheme{whale}
\beamertemplatenavigationsymbolsempty
\setbeamertemplate{footline}[frame number]
\usepackage{showexpl,xcolor}
\definecolor{light-gray}{gray}{0.95}


\lstloadlanguages{TeX}
\lstset{
  basicstyle=\ttfamily\small,
  breaklines=true,
  explpreset={numbers=none},
  language=[LaTeX]TeX,
  numbers=none,
  pos=r
}

% start with the actual content
\begin{document}
\title{Verwendungsbeispiele in \LaTeX}
\author{Hein-Erik Schnell, Sören Breidenbach, Thomas Rüd}

\frame{\titlepage}

\begin{frame}{Übersicht}
  \tableofcontents
\end{frame}

\section{Präsentationen}
\subsection{Was ist wichtig?}

\begin{frame}{Was kann \LaTeX?}{Präsentationen}
  \begin{itemize}
    \item[]<1-> \LaTeX bietet die Möglichkeit, Präsentationen zu erstellen.
    \item[]<1->
    \item[]<2-> Dabei ist wichtig:
      \begin{itemize}
        \item[]<3->
        \item<3-> Nicht zu viel Text
        \item<4-> übersichtlich gestalten
        \item<5-> Rechtschreibung
      \end{itemize}
  \end{itemize}
\end{frame}

\section{Grafiken}
\subsection{Paket graphicx}

\begin{frame}{Was kann \LaTeX?}{Grafiken}
  \begin{itemize}
    \item[]<1-> Mit dem Paket \textcolor{blue}{graphicx} kann man Bilder und
       Grafiken einfügen
    \item[]<1->
    \item[]<2-> Was kann \textcolor{blue}{graphicx}?
      \begin{itemize}
        \item[]<3->
        \item<3-> Größe der Bilder einstellen
        \item<4-> Bilder rotieren
        \item<5-> Beschriften
      \end{itemize}
  \end{itemize}

\end{frame}
\subsection{graphicx code-Beispiel}
\begin{frame}[containsverbatim,fragile]{Was kann \LaTeX?}{Code-Beispiel:
  graphicx}
  \begin{block}{Parameter von graphicx}
    \begin{LTXexample}[width=.3\linewidth]
      \includegraphics{tux}
      \includegraphics[scale=0.7,angle=45]{tux}
      \includegraphics[height=.8cm,width=1cm]{tux}
    \end{LTXexample}
  \end{block}
\end{frame}

% end of written content
\end{document}

% vim: ft=tex et sw=2 tw=80:
