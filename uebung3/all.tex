%%%names
% pdfTeX engine
% use KOMA class for cool presets like A4
\documentclass{scrartcl}

% load default packages for german and unicode support
\usepackage[ngerman]{babel}
\usepackage[T1]{fontenc}
\usepackage[utf8]{inputenc}

% load font to use (default linux libertine because it is a neat one
\usepackage{libertine}
\usepackage{libertinust1math}
\usepackage{booktabs}
\usepackage{float}
\usepackage{colortbl}
% start with the actual content
\begin{document}

In Tabelle \ref{Gemuse} werden wichtige Informationen über Gemüse präsentiert. Die einzigen Informationen, die man als erfolgreicher Landwirt nun noch braucht, ist die Kenntnis über die Wortlänge der Wochentage. Tabelle \ref{Tage} klärt darüber auf.
\begin{table}[t]
	\begin{tabular}{lrrrcc}
	
	\toprule
	Produkt & Herkunft & Saisonbeginn & Saisonende & Handelsklasse & verfügbar \\ \midrule
	Auberginen & Frankreich & Juli & September & I & - \\
	Esskastanien & Frankreich & September & September & I & - \\
	Feldsalat & Deutschland & Oktober & Februar & II & ja \\
	Kürbis & Deutschland & August & Dezember & I & ja \\
	Rote Beete & Italien & September & Februar & I & ja \\
	Zucchini & Spanien & Juni & Oktober & II & - \\
	Zwiebeln & Deutschland & Mai & Oktober & - & - \\
	\bottomrule
	
	\end{tabular}
\caption{Gemüse-Infos}
\label{Gemuse}
\end{table}

\begin{table}[h]
	\begin{tabular}{|>{\columncolor[gray]{.8}}l|
			>{\color{white}%
				\columncolor[gray]{.2}}l|
}
		Wochentag & Buchstaben\\
		
		Montag & 6\\
		Dienstag & 8\\
		Mittwoch & 8\\
		Donnerstag & 10\\
		Freitag & 7\\
		Samstag & 7\\
		Sonntag & 7\\
	\end{tabular}
\end{table}
% end of written content
\end{document}

% vim: ft=tex et sw=2 tw=80:
