% !TEX program = xelatex
% !TEX encoding = UTF-8 Unicode
% !TEX spellcheck = de-DE

% use KOMA class for cool presets like A4
\documentclass{scrartcl}

% load default packages for german and unicode support
\usepackage{polyglossia}
\setmainlanguage{german}
\usepackage{xltxtra}

% load booktabs for nice and pretty tables
\usepackage{booktabs}
\usepackage{multirow}

% set main font to our beloved libertine fonts
\setmainfont{Linux Libertine O}

% create new command to easily insert roman numbers, better handled by latex
% than just I, V etc. Command is called with \rom{SomeNumber}. In the background
% it actually does a bit more than just execute \romannumeral we make sure
% everything is stopped after the insertion with \relax. Then we first execute
% \romannumeral and make sure its uppercase afterwards. For that we use
% \expandafter.
% "Is this really necessary?", one might ask; probably not as \romannumeral
% by itself just replaces the number with an adequate letter like i v x c l ...
% BUT having numerals easily visible as such in the tex file is something that
% might be beneficial, furthermore it works for large numbers too, for which
% most dont remember the actual form to write in latin. Thus this command rocks!
% Further note: is \expandafter deprecated in latex: maybe
%               did I find a usuable alternative:    not really
\newcommand{\Urom}[1]{\uppercase\expandafter{\romannumeral #1\relax}}

% start with the actual content
\begin{document}
% start floating environment for good placement between paragraphs or at end
\begin{table}
  % start actual table with 2 left oriented columns one right and left with a
  % " - " between them and 2 more centered columns this enables fake align at
  % connection operator "-" between timespans as we use a multicolumn for the
  % column title and when the giving a span is not necessary.
  \begin{tabular}{llr@{ - }lcc}
    % add fancy horizontal line to mark start of table
    \toprule
    % this is the head for the columns
    Produkt     & Herkunft      & \multicolumn{2}{c}{Saison}
      & Handelskl. & verfügbar \\
    % thinner horizontal line to mark end of head for columns
    \midrule
    % table "body" content
    Auberginen  & Frankreich    & Juli          & September
      & \Urom{1}       & nein \\
    Esskastanien& Frankreich    & \multicolumn{2}{c}{September}
      & \Urom{1}       & nein \\
    Feldsalat   & Deutschland   & Oktober       & Februar
      & \Urom{2}       & ja \\
    Kürbis      & Deutschland   & August        & Dezember
      & \Urom{1}       & ja \\
    Rote Beete  & Italien       & September     & Februar
      & \Urom{1}       & ja \\
    Zucchini    & Spanien       & Juni          & Oktober
      & \Urom{2}       & nein \\
    Zwiebeln    & Deutschland   & Mai &         Oktober
      & --            & nein \\
    % add thick horizontal line to mark end of table
    \bottomrule
  % end actual table
  \end{tabular}
  % give the item in table environment (tabular) a name thats displayed below it
  \caption{Die vielsagende Tabelle}
  % give the float an internal name for references
  \label{tab:vielsagend}
\end{table}
% end of written content
\end{document}

% vim: ft=tex et sw=2 tw=80:
